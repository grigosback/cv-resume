\documentclass[10pt]{article}

\newcommand{\tab}{\hspace{\parindent}}
\renewcommand\refname{}

\usepackage{amsmath}
\usepackage{booktabs}
\usepackage{enumitem}
\usepackage{footnote}
\usepackage{fullpage}
\usepackage{hyperref}
\usepackage[margin=1.9cm]{geometry}

\begin{document}

{\raggedleft
August $23^{\text{rd}}, 2020$ \\
}
\noindent{\LARGE \textbf{Curriculum Vitae}}\\
\\
{\LARGE {José Ignacio Quinteros del Castillo}}\\

\noindent Date of birth: July $27^{\text{th}},\,\,1996$ \\
Place of birth: Salta, Argentina. \\
email: \url{jose.quinteros@ib.edu.ar} \\
Phone number: +54 9 387 523-9766

\section*{Academic formation}

\noindent \begin{tabular}{cl}

	Jul. 2017 - present	& \textbf{Telecommunications Engineering} \\
						& \tab Balseiro Institute, National University of Cuyo (UNCuyo) and National \\
						& \tab \tab Atomic Energy Commission (CNEA) - S. C. de Bariloche, Argentina. \\
						& \tab Final project \textit{System for emulation, acquisition and data processing for} \\
						& \tab \tab \textit{sensor matrices}. Telecommunications Engineering Department, \\
						& \tab \tab Centro Atómico Bariloche (CAB). \\
						& \tab Project advisor: Ing. Nicolás Catalano\,\footnotemark[1]. \\
						& \tab Average qualifications 8.06/10 \\

	Feb. 2014 - Jun. 2017	& \textbf{Electromechanical Engineering} (First 2 years) \\
						& \tab National University of Salta (UNSa) - Salta, Argentina \\
						& \tab To fulfill requirements for the Balseiro Institute selection process.\\
						& \tab Average qualifications 8.55/10 (34.15\% of the curriculum completed)\\

	Feb. 2009 - Dec. 2014	& \textbf{High school} \\ 
						& \tab Bachillerato Humanista Moderno secondary school - Salta, Argentina
\end{tabular}

\section*{Scholarships}

\begin{tabular}{cl}

Jul. 2017 - present 	& \textbf{Full scholarship for Telecommunications Engineering degree} \\
						& \tab From Balseiro Institute, National University of Cuyo (UNCuyo) and \\
						& \tab \tab National Atomic Energy Commission (CNEA). \\
						& \tab For being admitted as a Telecommunications Engineering student at \\
						& \tab \tab Balseiro Institute. \\
\end{tabular}

\section*{Skills and interests}

\hangindent=1.3cm \tab \textbf{Programming}\newline
C, Python, VHDL

\hangindent=1.3cm \textbf{Software}\newline
Xilinx Vivado \& SDK\\
MATLAB\\
KiCAD\\
Knowledge in simulation tools (CST, SPICE based simulators)

\hangindent=1.3cm \textbf{Networking}\newline
Cisco IOS\\
Sockets API

\hangindent=1.3cm \textbf{Languages}\newline
Spanish	\tab \tab \tab \tab \,\, Native\\
English \tab \tab \tab \tab \,\,\, Fluent\\
French, German \tab \tab Beginner

\hangindent=1.3cm \textbf{Interests}\newline
Communication systems design, signal processing, satellital communications, FPGAs, digital design.

\section*{Other activities}

\begin{tabular}{cl}

Mar. 2020 - present 	& \textbf{Telecommunications Engineering Carreer Commission} \\
						& \tab Academic Commission for the Telecommunications Engineering \\ & \tab \tab Carreer at Balseiro Institute\footnotemark[2]. \\
						& \tab Representative of the Student Body.\\		

Oct. 2017 - Apr. 2020 	& \textbf{Balseiro Institute Student Centre (CEIB)} \\
						& \tab Member of the Executive Commitee\footnotemark[3].\\
						& \tab Served as Secretary General and President.
\end{tabular}


\section*{References}

The following people may be contacted regarding the activities listed on this CV.

\begin{itemize}
	\item[] \footnotemark[1]\,Ing. Nicolás Catalano, Telecommunications Engineering Department, Centro Atómico Bariloche (CAB). Final project advisor. \url{nicolas.catalano@ib.edu.ar}
	\item[] \footnotemark[2]\,Dr. Juan Pablo Pascual, Director of the Telecommunications Engineering Carreer, Balseiro Institute.\\\url{juanpablo.pascual@ib.edu.ar}  
	\item[] \footnotemark[3]\,Executive Commitee, Balseiro Institute Student Centre (CEIB). \url{cdceib@gmail.com} 
\end{itemize}

\noindent The following links may also be of use
\begin{itemize}
	\item[] Balseiro Institute - Spanish version \url{www.ib.edu.ar}.
\end{itemize} 

\end{document}

